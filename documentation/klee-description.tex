\subsection{Description of KLEE Tool}
KLEE is a modern symbolic execution tool, capable of automatically generating test cases that 
achieve high coverage on a diverse set of complex and environmentally-intensive programs.
It can be used in a variety of ways to generate coverage tests, find bugs, and cross-check program
consistency.\cite{kleepaper}
KLEE works by running programs symbolically: unlike normal execution,
where operations along an execution path would produce concrete values from their operands,
KLEE will instead generate constraints that exactly describe the set of values possible
on that given path.
When KLEE detects an error or when a path reaches an exit call, 
KLEE will solve for the current path’s constraints (called its path condition)
and produce a test case where the program will follow the same path when rerun 
on an unmodified version of the checked program (complied without KLEE support).
\par
Our usage of KLEE in our project is to automatically generate tests cases for our
driver, with the maximum coverage possible.
This may be difficult in practice, as the driver uses Kernel API calls and other
runtime assumptions that may not exist if the code is extracted from the relevant code in
the Linux Kernel or referenced libraries required to run.
A test harness may have to be created around the code under test to ensure no issues
stop KLEE from doing what it needs to do.
\par
Once the test cases are generated, the test cases will be analyzed for any
potential exceptional behavior uncovered as well as for complete coverage of the 
anticipated scenarios the driver is supposed to run in.
A goal for coverage of our chosen driver would be 85\%, any less would not sufficiently
show that path analysis of the code would useful results in practice.
Coverage would be measured using the \textit{gcov} tool, which meaasures line coverage.
