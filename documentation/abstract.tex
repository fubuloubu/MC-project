Linux is the world’s most dominant operating system.
Between 2005 and 2015, over 11,800 individual developers from both private companies
and the general public contributed to the Linux kernel project.
The rate of change in the kernel has been historically high and continues to increase,
with over 10,000 patches going into each recent kernel release. \cite{linux}
However, with the ubiquity of use and the increasing nature of it's development
accelerating the project way past the ambitions of it's creator, Linus Torvalds,
it is becoming an extremely important task to ensure the correctness of the software.
Most importantly in the realm of security, as it's dominant role in the world's computing
systems means that any flaw can be exploited by malacious attackers.
No where in the codebase is this more imporant than in the hardware drivers present in
the operating system, as those drivers have elevated permissions in the kernel
that could easily allow an attacker to gain control of a system.
\par
Our project seeks to examine a moderately complex driver in the Linux Kernel 
using two different tools, klee \cite{klee} and LLBMC \cite{llbmc} in order to verify
correctness and freedom from bugs of the driver's code.
While not comparative in functionality, both tools are modern examples of Formal 
Verification tools that can be used to increase confidence in the correctness of software
through automated formal methods.
These tools contrast to more traditional methods of unit testing and functional testing
(or sometimes simply analysis by senior developers or static analysis tools)
typically employed on the drivers in the linux kernal, which may not be capturing the whole
story, or may be missing more complex failures of logic or concurrency which are hard to discover.
