\documentclass[10pt]{article}
\def\authorship{Bryant Eisenbach (bje2113) \\ Chandan Kanungo (ck2749)}
\usepackage[a5paper,margin=0.5in]{geometry}

%% Font
\usepackage[sfdefault,light]{roboto}
\fontsize{11}{14}

%% Header
\usepackage{fancyhdr,lastpage}
\usepackage[mmddyyyy]{datetime}
\renewcommand{\dateseparator}{/}
\fancypagestyle{plain}{
    \fancyhf{} %Clear Everything.
    \lhead{ \authorship }
    \chead{ CSEE E6863 }
    \rhead{ \jobname\ Page \thepage\ of \pageref*{LastPage} }
    \setlength{\headheight}{30pt}
    \setlength{\footskip}{24pt}
    \renewcommand{\headrulewidth}{1pt} %0pt for no rule, 2pt thicker etc...
    \renewcommand{\footrulewidth}{0pt}
}
\pagestyle{plain}

%% URLs
\usepackage[ocgcolorlinks]{hyperref}
\urlstyle{same}
\renewcommand\UrlFont{\color{blue}}

%% Lists
\usepackage[shortlabels]{enumitem}
 % Create environment that allows full-width text within list
\makeatletter
\newenvironment{fullwidth}
{\par%
    \setlength{\@totalleftmargin}{0pt}%
    \setlength{\linewidth}{\hsize}%
    \list{}{\setlength{\leftmargin}{0pt}}
\item\relax}
    {\endlist}
    \makeatother

%% Non-floating Captions
\usepackage{capt-of}

%% Tables 
\usepackage{booktabs}
\usepackage{longtable}
\usepackage{multirow}

%% CSV Files
\usepackage{csvsimple}

%% Graphs
\usepackage{pgfplots}
\usepgfplotslibrary{groupplots}
\pgfplotsset{width=10cm,compat=1.7}
\pgfplotsset{every axis legend/.append style={
             at={(0,0)},
                 anchor=north east}}

%% Code listings
\usepackage{minted}
\usepackage{listings}
\usepackage{color} % for colored solution
\usepackage{caption}
\setlength{\belowcaptionskip}{10pt plus 5pt minus 0pt}

% Needed to use \autoref with listings from minted
\providecommand*{\listingautorefname}{Listing}
% Command for referencing listings
\newcommand*{\listingref}[1]{\specialcell[r]{\nameref*{#1} \\ Source Code: \hyperref[{#1}]{\autoref*{#1}}}}

% This will allow us to use file names with
% underscores in below
\AtBeginDocument{
    \catcode`_=12
    \begingroup\lccode`~=`_
    \lowercase{\endgroup\let~}\sb
    \mathcode`_="8000
}

% Inline quick macros
\newmintinline[cpp]{cpp}{}
\newmintinline[ansic]{c}{}

% Custom macro for code listings from files
% #1 - language e.g. ocaml
% #2 - filename e.g. ../src/vislang.ml
% #3 - label e.g. src/top
% #4 - description e.g. Top Level
% note: filename will be pretty printed as vislang/src/*
%       instead of ../src/*
\newcommand{\srccode}[2]{
    \inputminted[linenos,fontsize=\footnotesize,frame=single,label=#2]{#1}{#2}
}

%% Maths
\usepackage{amsmath}

%% Patches (because this distro has an old version of LaTeX
\usepackage{fixltx2e}

%% BDDs
\usepackage{pgf}
\usepackage{tikz}
\usetikzlibrary{arrows,automata,shapes,positioning}
\tikzset{%
    zeroarrow/.style = {-stealth,dashed},
    onearrow/.style = {-stealth,solid},
    c/.style = {circle,draw,solid,minimum width=2em,
        minimum height=2em},
    r/.style = {rectangle,draw,solid,minimum width=2em,
        minimum height=2em}
}

%% Logic functions
\newcommand{\NOT}{\neg}
\newcommand{\OR}{\lor}
\newcommand{\AND}{\land}
\newcommand{\XOR}{\oplus}
\newcommand{\IMP}{\rightarrow}
\newcommand{\BDDAND}{\text{BDD\_AND}}


% CSEE 6863 Formal Verification HW/SW Systems
% Length: 1-2 pages, Due: October 27th
% Instructions:
%   Describe your project. What are the project goals? Include any relevant background.
%
%   Describe how your project relates to formal verification. In particular, explain how it 
%   relates to the main topics discussed in the course.
%
%   If your proposal includes any steps that may fail, such as using a tool on your own 
%   design or program, please include a plan-B (fallback) option. For example, a fallback 
%   option for using a tool on your own design could be extending a design that comes with the tool.
%
%   Use figures!
%
%   Important: You must disclose all sources that you intend to use. If your project is based on 
%   any of your prior work, it must be disclosed as well.

\begin{document}
    \section{Abstract}
    %DONE
    % Description of project
    Linux is the most widely use operating system in the world, 
as well as the largest open-source project in existance.
Between 2005 and 2015, over 11,800 individual developers from both private companies
and the general public contributed to the Linux kernel project.
The rate of change in the kernel has been historically high and continues to increase,
with over 10,000 patches going into each recent kernel release. \cite{linux}
\par
However, with the ubiquity of use and the increasing nature of its development
accelerating the project way past the ambitions of its creator, Linus Torvalds,
it is becoming an extremely important task to ensure the correctness of the software.
Due to its prevalence in many of the computing systems and mainframes in use world-wide,
the security of the codebase is of paramount importance to protecting Linux's billions 
of users, both directly and indirectly.
\par
Nowhere in the codebase is security more important than in the hardware drivers present in
the operating system, as those drivers have elevated permissions in the kernel
that could easily allow an attacker to gain control of a system.
Unfortunately, due to the difficulty of testing these drivers and the existence of
many different types of drivers for different hardware, this makes a wide vector
for potential intrusions, and a worthwhile focus for formal verification.
\par
Our project seeks to examine a moderately complex driver in the Linux Kernel 
using two different tools, Klee \cite{klee} and LLBMC \cite{llbmc} in order to verify
correctness and freedom from bugs of the driver's code.
While not comparative in functionality, both tools are modern examples of Formal 
Verification tools that can be used to increase confidence in the correctness of software
through automated formal methods.
These tools contrast to more traditional methods of unit testing and functional testing
%(or sometimes simply analysis by senior developers or static analysis tools)
typically employed on the drivers in the Linux kernel, which may not be capturing the whole
story, or may be missing more complex failures of logic or concurrency which are hard to discover.

    
    \section{Background}
    %DONE
    % Background about each tool
    \subsection{LLBMC is a static software analysis tool for finding bugs in  C and C++ program.The choice of LLBMC is, therefore, good as the Linux Kernel is written in C language.Also, the choice for this Bounded Model Checking tool is good in our case as it is mainly intended to check low-level system code.
The comparison of this Bounded Model Checking tool with the Klee tool would bring out some interesting observation on the differences between Bounded Model Checking and also the Symbolic Execution approach.
LLBM can achieve high test coverage ratios.Also, LLBM is powerful in a way because it supports all C construct including bitfields.It is a great tool to find memory access errors like heap and stack buffer overflows.LLBMC uses SMT solver as backend and therefore achieve higher performances.Its analysis results are closer to what actually runs on the machine.

In case LLBMC tool does not provide to be useful we are also considering CBMC tool.It is also a model checker for C and C++ programs.CBMC is GCC based not LLVM, so it would only be switched in case LLBMC won't provide useful results.

Also as an alternative is a change of driver from USB to some other driver or any equally complex program.
Also as an approach to test the tools first, a bug known program would be used and then the main experiment on the USB driver.}

    \subsection{Description of KLEE Tool}

    
    %DONE
    % Background about software component under test
    \subsection{Design Under Verification}
Based on prior work\cite{usbkbd}, we intend on verifying a section of the Linux kernel
that contains kernel modules related to USB Human Input Devices (HID).
These modules are used to interact with HID devices in order to provide a bridge layer
to the input kernal module.
This area of kernel design is an important design to test as it is an active vector
for physical attacks on a computing system.
\par
We intend to test usbmouse.c first \cite{usbhid}, 
which is a complex program of just over 200 sloc that interacts with the kernelspace using
the Linux Kernel API.
Per the prior work, we see that there is a need to build a framework around this kernel module
in order for it to instantiate properly, so work will need to be done to properly construct
this framework.
\par
% DONE: Sources and licensing for program under test,
%       as well as tools being tested
The Linux Project and all contained software modules are governed under the GNU General
Public License (GPL), which states that the source is freely available to copy and/or change
for the purposes of this project, with the restriction that any work using the project
must not contain further restrictions and the source code must be made available.
We are distributing this project on GitHub, thereby fulfilling the relevant clauses of the GPL.

    
    \section{Project Description}

    \subsection{Project Goals}
    %DONE
    % Describe project goals
    The project seeks to test the driver mentioned above in a relevant context of proving
    that the software module meets the intended design parameters and is free of all bugs
    being tested, including concurrency and usage errors.
    We will do this by employing the two tools mentioned in order to provide as complete
    a basis as possible for identifying the software as correct.
    The comparison of the LLBMC tool with the Klee tool should allow some interesting 
    observations on the differences between Bounded Model Checking
    and the Symbolic Execution approach, and which kinds of bugs or programs may be found.
    
    \subsection{Alternative Paths}
    %DONE
    % Describe plan B
    If we are unable to create a meaningful framework for testing the driver, 
    we will resort to testing a piece of software developed by the team.
    One choice is a compiler that one of the team members (Bryant Eisenbach) has developed
    for COMS W4115 (Programming Languages and Translators) in Summer 2015 using the OCaml
    programming language (which has LLVM bindings for a compilation target).
    The use of this project is acceptable as it contains 1300 sloc and can compile to LLVM,
    which can be used by at least one of the tools.
    \par
    In case LLBMC or Klee tool does not provide to be useful we are also considering CBMC tool.
    It is a bounded model checker for C and C++ programs.
    However, CBMC is GCC-based (not LLVM), so switching to CBMC may prove more difficult 
    for our intended usage scenario.
    
    %DONE: Insert Bibliography
    \pagebreak
    \small
    \bibliography{bibliography.bib}{}
    \bibliographystyle{plain}

\end{document}
